\section{Conclusions}
In this paper, we model the traffic signal couplings in a traffic system, where continuous signals from various channels and nodes interact and couple with each other in terms of temporal, spatial and spatio-temporal aspects of traffic conditions. They are modeled by a novel deep graph network MS-GAT, which characterizes multi-dimensional, time-evolving and multi-relational traffic interactions in the road network and forecast future traffic conditions from historical spatio-temporal traffic signals. MS-GAT explicitly models the latent relations between diversified traffic signals in terms of \textit{channel relations} and integrates them  with \textit{spatial} and \textit{temporal relations} in a synchronous way into the neural traffic graph for forecasting. These are modeled by (1) a module MEAM stacking replicas of deep learning architectures to capture the complicated spatio-temporal dynamics of traffic conditions and their multi-relations and influence on future traffic conditions, and (2) a multi-component structure TPCs that adopts a time gated fusion mechanism to adaptively focus on the traffic conditions at different past stages. Further, MS-GAT maintains a small number of parameters with a convenient multi-dimensional self-attention scheme applicable to any data with multi-dimensional features. We substantially test MS-GAT on five real-world datasets against seven state-of-the-art graph neural networks for traffic prediction, which show that MS-GAT outperforms the baselines. 

MS-GAT is a general spatio-temporal forecasting framework, applicable to other spatio-temporal structured sequence forecasting scenarios, such as preference prediction in recommender systems and air quality forecasting, since it is characterized by providing a multi-view representation for each node in graphs and a multi-relational representation for graph-based complex systems. In ITS, in the case of the data from different traffic modes (e.g., bus, private car, bike) or external modes (e.g., passengers or weather) in an identical traffic region is available, MS-GAT can also attempt to handle these multi-modal data and their interactions through treating each modal data as a separate information channel to manipulate. We argue that there exist some potential causal associations between those available multi-modal data and expectations in traffic prediction tasks, such as the impact of current rainfall on future traffic flow in a road network, whereas MS-GAT just can deal with that. Besides, MS-GAT can be also expanded in the following directions. First, we will further optimize the network structure and parameters to improve the prediction accuracy on sharp changes of traffic conditions that may be induced by external factors, e.g., weather or abnormal events. Second, like other spatio-temporal forecasting methods, we will also explore the mechanism of modeling the spatio-temporal dependencies for evolving graph structures.