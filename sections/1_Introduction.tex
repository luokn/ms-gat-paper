\section{Introduction}
\label{sec:introduction}

\IEEEPARstart{D}{ata}-driven traffic prediction  \cite{vlahogianni2014short} forms a critical task of intelligent transport systems (ITS), valuable for various real-world applications such as efficient traffic management, dynamic route planning, and intelligent guidance service. Traffic prediction estimates the future traffic conditions based on historical observations of traffic systems such as physical road networks. Traffic conditions are reflected on temporal and spatial traffic signals such as channel or node-based traffic volume, density, speed, and others, captured by diverse sensors deployed at different geospatial locations in a traffic network. All aspects of traffic signals are coupled with each other temporally and spatially, forming a \textit{coupled traffic network}, similar to many other complex systems where objects and their attributes interact and couple with each other \cite{2015Coupling,WangGC21}. For example, the traffic conditions at one traffic node may be associated with those at its neighbouring and farther nodes before and after along the traffic network, forming \textit{spatial relations} between traffic network nodes. Second, the traffic signals of one node may be temporally related to those of backward and forward nodes, forming \textit{temporal relations}. Both spatial and temporal relations are node- and time-evolving, the same as their traffic conditions. They are further subject to the burst of traffic accidents and special events taken place on the way. Further, the spatial and temporal relations are interdependent, forming \textit{spatio-temporal relations} in traffic systems. In addition, similar to the user-item interations in recommender systems \cite{cao2016non}, the spatio-temporal interactions in a traffic network and among its traffic conditions may be explicit and implicit, and hierarchical and heterogeneous \cite{zhu2020unsupervised}. These multi-aspect interactions and couplings make it difficult to precisely predict how the rise of vehicle flow at one node at a time point may cause the change of vehicle flows at other nodes at future time steps. It is thus essential yet challenging to effectively model the various spatio-temporal traffic signal couplings over channels and nodes, which could better capture the intrinsic characteristics and complexities in traffic networks, leading to better traffic modeling and prediction.

Extensive data-driven methods have been studied to analyze some relations in traffic systems in a road network, which can be categorized into time-series methods, shallow machine learning methods, and deep learning methods. First, in the early stage, traditional time-series methods such as  historical average model \cite{liu2004summary}, auto regressive integrated moving average (ARIMA) \cite{williams2003modeling}, Kalman filtering model \cite{OKUTANI19841} and canonical  vector auto regressions (VAR) \cite{zivot2006vector} were widely used in traffic forecasting. Then, shallow machine learners such as Support Vector Regression (SVR) \cite{1364002}, Bayesian model \cite{1603558} and $k-$nearest neighbor method \cite{zhang2009short} succeeded. Lastly, deep neural networks (DNNs) has dominated today's literature showing the state-of-the-art performance of traffic prediction by capturing sequential relations in a large amount of traffic sequential data \cite{lv2014traffic} and \cite{yu2017deep}. For example, temporal dependencies are modeled in \cite{zhao2017lstm} and \cite{ma2015long} by recurrent neural networks (RNNs). In \cite{wu2016short,ma2017learning,zhang2017deep}, spatial dependencies are captured by convolutional neural networks (CNNs). Both RNNs and CNNs are insufficient in modeling spatio-temporal traffic interactions, which are more successfully characterized by the recent advances in graph neural networks (GNNs). GNNs present the interactions between entities as a graph and can capture spatial node interactions in a road network \cite{wu2020comprehensive, jiang2021graph}. Recent methods including STGCN \cite{yu2017spatio}, DCRNN \cite{li2017diffusion}, Graph WaveNet \cite{wu2019graph} and AGCRN \cite{bai2020adaptive} essentially formulate the traffic prediction as a spatio-temporal graph modeling problem and achieve more superior performance in traffic forecasting. GNN-based models focus on optimizing the representation of either spatial relations (e.g., \cite{yu2017spatio} in urban roads by graph convolutional networks (GCN) and in \cite{li2017diffusion} based on diffusion convolutional networks) or temporal relations (e.g., learning temporal patterns in \cite{bai2020adaptive} and \cite{wang2020traffic} by gated recurrent units (GRU)).

Though DNNs including GNNs achieve the state-of-the-art traffic prediction performance by applying increasingly advanced network architectures and learning mechanisms and overparametering the networks, the current research focus is mainly on characterizing latent features and relations in a traffic network, e.g., the above spatial, temporal, and spatio-temporal features or relations. By taking a coupled traffic netowrk view with multi-aspect traffic signals coupled, there are still various issues and data complexities yet explored or insufficiently characterized. For example, sensors deployed at a node capture various aspects of traffic signals such as speed, volume and density, which are coupled in reality and thus their relations should be jointly represented to capture the multi-view node conditions. Further, the traffic conditions at one node or on one road may be related to that at other nodes or roads, i.e., a signal at a node and time point is essentially embedded in the entire traffic system and its dynamics. This coupled traffic network view focuses on modeling rich signal couplings and interactions in traffic systems, capable of addressing problems such that how an accident at one point may not only cause problems to its node and neighboring nodes and roads but also to other places affecting the evolving traffic network.

Specifically, building on the stronger capability of GNNs and addressing their significant gaps in modeling complex traffic systems, this work models the above multiple aspects of signal couplings and their evolution explicitly or implicitly embedded in complex traffic networks. We introduce a framework of modeling spatio-temporal traffic signal couplings (see Fig.~\ref{fig:core}), followed by a novel spatio-temporal graph network (see Fig.~\ref{fig:framework}), named Multi-relational Synchronous Graph Attention Networks (MS-GAT). In MS-GAT, similar to the concept of image channel in computer vision, each traffic signal is viewed as a measurement channel at a traffic node, where the temporal signal movement over time forms the temporal channel representation; each node is often characterized by signals collected from multiple sensors deployed on the spot, with the signals interacting with each other to capture the multi-channel interdependencies; nodes in the traffic network are further connected to capture their spatial relations. Consequently, MS-GAT captures temporal relations in each signal over time, spatial relations over nodes, and multi-channel relations between signals. These are jointly modeled for traffic prediction. By viewing traffic systems as coupled traffic networks with coupled traffic signals over sensors, our contributions are as follows:

i) We recognize the significance of a kind of latent relation (called \textit{channel relation} in this paper) and characterize this noteworthy relation in our proposed spatial-temporal graph model. To the best of our knowledge, this is the first work on explicitly modeling this channel relation in traffic prediction task.

ii) We propose a concrete spatial-temporal graph model based on multi-component fusion, called MS-GAT, which is accompanied by a flexible and effective data augmentation scheme for its supervised learning process. In this model, the newly-attended channel relation together with the other two familiar relations are picked abreast up by a core module called multi-relational embedding abreast module (MEAM). Furthermore, their contributions to the resulted traffic conditions in real-world forecasting scenarios are adaptively distinguished in a synchronous manner by a developed deep framework based on our devised multi-dimensional self-attention scheme.

iii) We develop a simple yet effective multi-dimensional self-attention scheme to improve the conveniences of applying and implementing self-attention mechanism for handling multi-dimensional input data, which is also interesting in ameliorating the influence of attention mechanism on model size while ensuring its prediction capability.

We conduct a series of experiments on five real-world traffic datasets. Experimental results demonstrate the superior performance of the proposed MS-GAT\footnote{The source code is publicly available from \url{https://github.com/luokn/ms-gat}}.
The remainder of this paper is organized as follows. Section II reviews the related work. The preliminaries are presented in Section III. Section IV details the proposed approach. Section V reports the experimental results, followed by the discussion in Section VI. Section VII concludes the paper.
